%% ========================================================================
%%							Abstract
%% ========================================================================


\chapter*{Kurzfassung}
\label{cha:kurzfassung}


Versicherungsunternehmen sehen sich mit einer immer größer werdenden Anzahl von Vorschriften konfrontiert. Ein Teil dieser Richtlinien beschäftigt sich dabei mit der Fragestellung wie Versicherungsportfolios in die Zukunft projiziert werden sollen. Bedenkt man dabei, dass ein durchschnittliches Versicherungsunternehmen hunderttausende von Polizzen im Bestand hat, kann man den Zeitaufwand erahnen, der für solche Prognoserechnungen aufzuwenden ist. Steigende Bestände von Versicherungspolizzen in Kombination mit engen aufsichtsrechtlich gesetzten Fristen führen dazu, dass Versicherungsunternehmen nach Möglichkeiten suchen, die Rechenzeit für Projektionen zu verkürzen. In den letzten Jahren wurden bereits verschiedene Ansätze entwickelt und zum Einsatz gebracht, um ähnliche Polizzen zusammenzufassen und damit die Rechenzeit zu verkürzen. Dabei reichen die derzeit verwendeten Methoden von einfachen Zusammenfassungen von Polizzen mit exakt gleichen Attributen bis hin zu Clusteralgorithmen wie beispielswiese k-means. Diese Arbeit zeigt mögliche Probleme mit den derzeit verwendeten Algorithmen auf und präsentiert neue Implementierungsansätze aus dem Bereich des maschinellen Lernens, um eine Gruppierung des Versicherungsbestandes noch effektiver umsetzen zu können.
