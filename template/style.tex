%% ========================================================================
%%								Fabian Pribahsnik
%% ========================================================================
%%
%% ========================================================================
%%						All different kinds of style settings and other stuff
%% ========================================================================

\documentclass[% KOMA Script
fontsize=\myfontsize,%% size of the main text
paper=\mypapersize,  %% paper format
parskip=\myparskip,  %% vertical space between paragraphs (instead of indenting first par-line)
DIV=calc,            %% calculates a good DIV value for type area; 66 characters/line is great
%headinclude=true,    %% is header part of margin space or part of page content?
headinclude=false,
footinclude=false,   %% is footer part of margin space or part of page content?
open=right,          %% "right" or "left": start new chapter on right or left page
appendixprefix=true, %% adds appendix prefix; only for book-classes with \backmatter
bibliography=totoc,  %% adds the bibliography to table of contents (without number)
draft=\mydraft,      %% if true: included graphics are omitted and black boxes
                     %%          mark overfull boxes in margin space
%BCOR=\myBCOR,        %% binding correction (depends on how you bind
                     %% the resulting printout.
\mylaterality        %% oneside: document is not printed on left and right sides, only right side
                     %% twoside: document is printed on left and right sides
]{scrbook}  %% article class of KOMA: "scrartcl", "scrreprt", or "scrbook".
            %% CAUTION: If documentclass will be changed, *many* other things
            %%          change as well like heading structure, ...
            
\usepackage[utf8]{inputenc} %% UTF8 as input characters
\usepackage[ngerman,american]{babel}  %% used languages; default language is *last* language of options
\usepackage{csquotes}	%% quotations
\usepackage{scrpage2} %%  advanced page style using KOMA
\usepackage{graphicx}
%\usepackage[pdftex]{graphicx}%% The widely used package to use graphical images within a \LaTeX{} document.
\usepackage[\mytodonotesoptions]{todonotes}  %% option "disable" removes all todonotes output from resulting document
\usepackage{enumitem} %% enables the user to customize the enumerate,.. (make lists with a), b), c) or A), B), ....)
\usepackage{amsmath} %% math environment for equations
\usepackage{amsthm}   %% math environment for theorems, definitions,...
\usepackage{amssymb} %% blackboard bold symbols for expectation,...
\usepackage{algorithm} % environment to write pseudo code 
\usepackage[noend]{algpseudocode}% environment to write pseudo code 
\usepackage{xifthen}    %% makes it possible to make the output of a new command dependent on the input values for that command. (see articlequote)
\usepackage{tabu, booktabs, makecell} %%tabu: make tables within the text; booktabs: vertical lines; makecell: multiline cells
\usepackage[labelfont=bf]{caption}%%needed to reference on tables and figures make Table,... bold
\usepackage{actuarialangle} %%needed for the annuity-symbol
\usepackage{eurosym} %% get the euro symbol
\usepackage{pdflscape} %% rotate pages to landscape (e.g. big tables)
\usepackage{afterpage} %% clear all floating objects (used for pdflscape)
\usepackage{enumitem} %% for roman enumeration
\usepackage{bbm} %% make the math symbol for Indicator 1 (\mathbbm{1})
\usepackage{listings} %% implemnt R code 


\usepackage[backend=biber, %% using "biber" to compile references (instead of "biblatex")
style=\mybiblatexstyle, %% see biblatex documentation
%dashed=\mybiblatexdashed, %% do *not* replace recurring reference authors with a dash
backref=\mybiblatexbackref, %% create backlings from references to citations
natbib=true, %% offering natbib-compatible commands
hyperref=true, %% using hyperref-package references
]{biblatex}  %% remove, if using BibTeX instead of biblatex

\addbibresource{\mybiblatexfile}			%% Name of the biblatex file that holds the references.

%define colours
\definecolor{backcolour}{rgb}{0.95,0.95,0.92}
\definecolor{codepurple}{rgb}{0.58,0,0.82}

\lstset{%Settings for how I want the code included
	language=R,
	basicstyle=\scriptsize\ttfamily,
	commentstyle=\ttfamily\color{blue},
	stringstyle=\color{codepurple},
	keywordstyle=\color{magenta},
	stepnumber=1,
	numbersep=5pt,
	backgroundcolor=\color{backcolour},
	showspaces=false,
	showstringspaces=false,
	showtabs=false,
	numbers=left,
	frame=single,
	tabsize=2,
	captionpos=b,
	breaklines=true,
	breakatwhitespace=false,
	escapeinside={},
	morekeywords={}
}