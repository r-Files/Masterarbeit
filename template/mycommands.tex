%% ========================================================================
%%							mycommands
%% ========================================================================

\newcommand{\articlequote}[3]{
	\begin{quotation}
		\begin{center}
			\textit{#1}
			% \ifthenelse{〈test expression〉}{〈true code〉}{〈false code〉}
			% \equal{〈string1〉}{〈string2〉}
			\ifthenelse{\equal{#2}{}}{}{\\ \medskip \textbf{#2}}
			\bigskip
		\end{center}
	#3
	\end{quotation}
}

\newcommand{\E}{\mathbb{E}}
\newcommand{\R}{\mathbb{R}}
\newcommand{\V}{\mathcal{V}}


% define new maths operators 
\DeclareMathOperator*{\argmin}{arg\,min} % thin space, limits underneath in displays

% used for enumeration styles
\newcommand{\subscript}[2]{$#1 _ #2$}

% environments for definition, theorems, remarks,...
\newtheorem{definition}{Definition}[chapter]
\newtheorem{corollary}{Corollary}[chapter]
\newtheorem{lemma}{Lemma}[chapter]
\newtheorem{example}{Example}[chapter]
\newtheorem{remark}{Remark}[chapter]

% draw right angle with tikz
\newcommand*{\rightangle}[3]{% #1 = point, #2 = start angle, #3 = radius
	\draw[shift={(#2:#3)}] (#1) arc[start angle=#2, delta angle=90, radius = #3];
	\fill[shift={(#2+45:#3/2)}] (#1) circle[radius=1.25\pgflinewidth];
}

% make the everything dependent --> 1 counter for all 
	%\newtheorem{corollary}[definition]{Corollary}
	%\newtheorem{lemma}[definition]{Lemma}
	%\newtheorem{example}[definition]{Example}