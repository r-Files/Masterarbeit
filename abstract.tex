%% ========================================================================
%%							Abstract
%% ========================================================================


\chapter*{Abstract}
\label{cha:abstract}


Insurance companies are facing a huge amount of regulations, including various guidelines addressing forecast scenario calculations for the policies in the portfolio. Taking the hundreds of thousands policies into account an average insurance company has in its portfolio on can easily see that these scenario calculations are very time consuming. Due to the rising number of policies and the very tight time schedule introduced with Solvency II insurance companies are looking for ways to reduce the computational time significantly. In the past years different approaches were developed and already used for grouping similar policies together and therefore reducing the computation time. The currently used algorithms are ranging from just grouping policies with exactly the same attributes together to basic cluster algorithms like $k$-means. This work highlights potential problems with the algorithms currently used and shows the implementation of some machine learning techniques which can be used to replicate cash flows of insurance policies. 
