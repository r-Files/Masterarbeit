%% ========================================================================
%%							Conclusion
%% ========================================================================


\chapter{Conclusion}
\label{cha:Conclusion}

The aim of this work was to analyse procedures that allow for a compression of an insurance portfolio. A representative subset of policies should be selected in such a way that projected cash flows can be replicated as accurately as possible. A special focus was placed on the high dimensionality of the data and the associated challenges. The $k$-means algorithm presented in chapter 3 was particularly convincing due to its simplicity of implementation, but also had weaknesses. Especially in the case of high-dimensional property spaces, the curse of dimensionality and the associated difficulty in determining the optimal number of cluster centres must be emphasized. The non-negative least squares algorithm presented in Chapter 4 is also well established in practice and is therefore used in insurance companies.  However, numerical instabilities must be pointed out when using this method and implementations based on inversions of matrices are strongly discouraged in this context. Instead, it could be shown that by applying $QR$-decompositions, significantly more robust results could be achieved and therefore those approaches are to be preferred. In the last chapter, neural networks were presented as a method that has so far received little attention in the insurance sector. With a network consisting of less than 100 neurons, it is possible to replicate cash flows over a period of 60 years so well that the deviation is significantly below 2\%. Even if these neuronal networks cannot yet be used to compress portfolios, the potential applications are interesting. For example, such trained networks could be used to quickly forecast estimated cash flow patterns for entire portfolios. The training effort could be carried out in advance in order to receive first results within minutes after the delivery of the monthly policy data. A more complex use case of neural networks would be that in which neural networks are used to estimate important parameters in the context of solvency calculations. Initial attempts in this direction are promising and require further in-depth analysis \cite{slides}.