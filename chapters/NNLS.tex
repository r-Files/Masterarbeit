%% ========================================================================
%%							NNLS
%% ========================================================================


\chapter{Non-negative least squares (NNLS)}
\label{cha:NNLS}

Another way to find a grouped portfolio that minimizes the deviation defined in formula (\ref{eq:objective_function}) is to solve the problem using mathematical optimization methods. The goal is to find a subset of the portfolio and scale it in way so that the square deviation becomes minimal. Of course, this approach must ensure that scaling is only possible in the positive direction. This means that it makes no sense to have a negative policy in the grouped portfolio, because it cannot be defined, not to mention explained logically. 

\begin{definition}[Non-negative least squares]\label{def:NNLS}
	Let $P \subset \V$ be a portfolio, $A \in \R^{m \times n}$ the matrix with the corresponding cashflows and $b \in \R^m$ the vector with the summed cashflows. Then $x \in \R^n$, the vector of scaling, should be optimized such that:  
	\begin{equation}\label{eq:NNLS}
		\begin{aligned}
			\argmin_x \lVert Ax -b \rVert_2^2 \\
			\text{subject to } x \geq 0
		\end{aligned}
	\end{equation}
\end{definition}

\begin{remark}
	\leavevmode % needed for items to start in new line after remark.
	\makeatletter
	\@nobreaktrue
	\makeatother
	\begin{itemize}
		\item 	The entries of the vector $x$ are the so-called scaling values for the cashflows in matrix $A$. Each entry $x_i, i \in \{1,...n\}$ that is greater than zero scales to the $i$-th column of matrix $A$ where column $i$ represents the cashflows of the $i$-th policy of the portfolio.
		\item 	It should be noted that this approach optimizes cashflows, not policies. Since there is a one-to-one relationship between cashflows and policies, the scaling factors cannot only be used to scale the cashflows but also to scale the policies in order to create a grouped portfolio. 
		\item 	Scaling policies can also involve risks, as a policy that is scaled by a factor of 2 will not necessarily produce cash flows that are also increased by a factor of 2. This can be attributed to the fact that, for example, non-linearities occur due to discount effects for higher premiums in the tariff.
	\end{itemize}
\end{remark}

\begin{remark}
	The number of policies in the grouped portfolio corresponds exactly to the number of entires greater than zero in the vector $x$.
\end{remark}

One of the well known algorithms for solving the non-negative least square problem is that of Lawson and Hanson, which uses an active set method. The steps necessary for solving that problem are given in \cite{lawson}. Additionally to those parameters defined in definition \ref{def:NNLS} one also needs a real value variable $\epsilon$ as a stopping criterion.

\begin{algorithm}
	\caption{Non-negative least squares \cite{lawson}}\label{alg:NNLS}
	\begin{algorithmic}
		\\
		\begin{enumerate}
			\item Set $P = \emptyset$, $R = \{1, ..., n\}$, $x$ = $0_{n \times 1}$
			\item Compute $w = A^\top(y - Ax)$.
			\item While $R \neq \emptyset$ and $max(w) > \epsilon$
			\begin{enumerate}[label=\emph{\alph*})]
				\item Find index $j \in R$ such that $w_j = max\{w_t, t \in R\}$.
				\item Move the index $j$ from $R$ to $P$.
				\item Let $A^P$ be $A$ restricted to the variables included in $P$.
				\item Let $s$ be a vector of same length as $x$. Let $s^P$ denote the sub-vector with indexes from $P$, and let $s^R$ denote the sub-vector with indexes from $R$.
				\item Compute $s^P = ((A^P)^\top A^P)^{-1} (A^P)^\top y$
				\item Set $s^R = 0$.
				\item While $min(s^P) \leq 0$
				\begin{enumerate}
					\item Set $\alpha = min\frac{x_i}{x_i - s_i}$ or $i$ in $P$ where $s_i \leq 0$
					\item Set $x = x + a(s-x)$
					\item Move from $P$ to $R$ all indices $j \in P$ for which $x_j = 0$.
					\item Compute $s^P = ((A^P)^\top A^P)^{-1} (A^P)^\top y$
					\item Set $s^R = 0$
				\end{enumerate}
				\item Set $x$ to $s$
				\item Compute $w = A^\top(y - Ax)$.
			\end{enumerate}
		\end{enumerate}
	\end{algorithmic}
\end{algorithm}

Algorithm \ref{alg:NNLS} consists, apart from the initialization, of a main loop and an inner loop. The loops are highlighted by indentations and start at point 3 and point g) respectively. If for a variable reference is made to those indices of the variable which are contained in the set $R$, then these entries are 0. All indexes that exist in the set P, by contrast, have nonzero values. If such a variable has a negative value, the algorithm either moves it to the positive value range or sets it to zero. By setting a variable to zero, the index is also shifted from the set $P$ to the set $R$. This ensures that the following condition is met at the end of the algorithm.   
\begin{align*}
	x_j &> 0, \quad j \in P \\
	x_j &= 0, \quad j \in R
\end{align*}



\begin{remark}\label{rem:gradient}
	Let $f(x) = \lVert Ax -b \rVert_2^2$ be the function, then the gradient is given by: 
	\begin{equation*}
		\nabla f(x) = \lVert Ax -b \rVert_2^2 = A^\top(Ax-b)
	\end{equation*}
	
	\begin{align*}
		\nabla \lVert Ax -b \rVert_2^2	&= \nabla (Ax-b)^\top (Ax-b) \\
										&= \nabla (x^\top A^\top - b^\top)(Ax-b) \\
										&= \nabla (x^\top A^\top Ax - x^\top A^\top b - b^\top Ax + b^\top b) \\
										&= \nabla (x^\top A^\top Ax - 2x^\top A^\top b + b^\top b) \\
										&= 2 (A^\top A x - A^\top b) \\
										&= 2A^\top (Ax - b)
	\end{align*}
\end{remark}

As shown in remark \ref{rem:gradient}, step 2 of algorithm \ref{alg:NNLS} calculates the negative gradient of the ordinary least squares problem. In the next step it is checked whether the inner loop still has to be executed or not. If the index set $R$ correspond to the empty set then all indexes are already in $P$ which means that all entries of $x$ are positive. This case is not desirable, as no compression can be achieved. If $max(w) \leq \epsilon$ is satisfied, the gradient has no entry large enough so that an substantial improvement can be achieved and the algorithm has reached the optimum. 









