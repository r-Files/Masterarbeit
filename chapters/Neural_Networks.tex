%% ========================================================================
%%							NNLS
%% ========================================================================


\chapter{Neural Networks (NN)}
\label{cha:NN}

In the last chapter of this thesis, a mathematical concept is discussed that has attracted much attention in the recent past, namely neural networks. The area of artificial intelligence (AI) in which neural networks are embedded has been the subject of an intense media hype in recent years. In 2016, for example, a computer program developed by the British company Google DeepMind succeeded for the first time in defeating Lee Sedol of South Korea, considered to be the strongest Go player in the world \cite{wiki_01}. This victory of a machine against a human being is considered to be a milestone in the field of artificial intelligence \cite{LA_Times}. Further successes were also achieved in the area of real-time games at the beginning of 2019. For the first time DeepMind's AlphaStar was able to defeat the world's best players in StarCraft, considered to be one of the most challenging real-time strategy games \cite{AlphaStar}. Also Gartner, a global research and advisory firm, which publishes the well-known but definitely criticisable hype cycle representations considers AI as one of the most important technologies of recent years. In the hype cyle for emerging technologies from 2017 there are 4 out of the 32 listed technologies that can be attributed to the field of AI, such as Deep Learning or Machine Learning \cite{Gartner2017}. Also in the following years 2018 and 2019 technologies that clearly belong to the AI sector were mentioned in the hype cycle 5 and 6 times respectively \cite{Gartner2018}, \cite{Gartner2019}. It is therefore clear to see that on the one hand the technology behind AI has enormous potential to solve problems that have so far been assumed to be solved only by humans. On the other hand, it can be assumed that this trend is not just a short-lived phenomenon, but a continuous process leading to a transformation to AI-based systems. Therefore, it is even more important to understand the functionalities of these technologies and to discuss their applicability to the insurance industry. Whether in the end a completely automated grouping algorithm at all is possible or can be implemented with the available resources remains open. However, the aim is to provide an overview of the basic functionality and to present case studies with insurance data. 



Since various terms and buzzwords related to artificial intelligence are often used differently in media reports, the following paragraph provides a general framework for a better understanding.



\cite{wiki_01}



\cite{Allaire2018}


Another way to find a grouped portfolio that minimizes the deviation defined in formula (\ref{eq:objective_function}) is to solve the problem using mathematical optimization methods. The goal is to find a subset of the portfolio and scale it in way so that the square deviation becomes minimal. Of course, this approach must ensure that scaling is only possible in the positive direction. This means that it makes no sense to have a negative policy in the grouped portfolio, because it cannot be defined, not to mention explained logically. 

\begin{definition}[Non-negative least squares]\label{def:NNLS}
	Let $P \subset \V$ be a portfolio, $A \in \R^{m \times n}$ the matrix with the corresponding cashflows and $b \in \R^m$ the vector with the summed cashflows. Then $x \in \R^n$, the vector of scaling, should be optimized such that:  
	\begin{equation}\label{eq:NNLS}
		\begin{aligned}
			\argmin_x \lVert Ax -b \rVert_2^2 \\
			\text{subject to } x \geq 0
		\end{aligned}
	\end{equation}
\end{definition}

\begin{remark}
	\leavevmode % needed for items to start in new line after remark.
	\makeatletter
	\@nobreaktrue
	\makeatother
	\begin{itemize}
		\item 	The entries of the vector $x$ are the so-called scaling values for the cashflows in matrix $A$. Each entry $x_i, i \in \{1,...n\}$ that is greater than zero scales to the $i$-th column of matrix $A$ where column $i$ represents the cashflows of the $i$-th policy of the portfolio.
		\item 	It should be noted that this approach optimizes cashflows, not policies. Since there is a one-to-one relationship between cashflows and policies, the scaling factors cannot only be used to scale the cashflows but also to scale the policies in order to create a grouped portfolio. 
		\item 	Scaling policies can also involve risks, as a policy that is scaled by a factor of 2 will not necessarily produce cash flows that are also increased by a factor of 2. This can be attributed to the fact that, for example, non-linearities occur due to discount effects for higher premiums in the tariff.
	\end{itemize}
\end{remark}


